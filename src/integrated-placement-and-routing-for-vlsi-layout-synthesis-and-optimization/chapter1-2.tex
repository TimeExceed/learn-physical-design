\documentclass[UTF8,lualatex]{ctexbeamer}
\usepackage{graphbox}
\usepackage{tabu}
\usepackage{amsmath}
\usepackage{algorithmicx}
\usepackage[noend]{algpseudocode}

\usetheme{Frankfurt}
\setbeamertemplate{note page}[plain]
% $BEAMER_NOTES

\title{\kaishu Placement Using Efficient 2-D Compaction}
    \author{Ping-San Tzeng\\
        slided by 陶大}
    \date{2021-11-19}

\newcommand{\hpwl}{\text{HPWL}}
\newcommand{\nets}{\text{nets}}
\newcommand{\cells}{\text{cells}}
\newcommand{\edges}{\text{edges}}
\newcommand{\pads}{\text{pads}}


\begin{document}
\songti
    
\begin{frame}
\titlepage
\end{frame}

\section{The Problem}

\begin{frame}
    \frametitle{VLSI Placement Problem}

    \begin{alertblock}{问题}
        \begin{itemize}
            \item 非常多的cells (e.g., $10^{10}$)
            \item route尽可能短的线/塞进非常小的面积
            \item 算法尽可能地快
        \end{itemize}
    \end{alertblock}
    \begin{exampleblock}{之前产业界的做法:std cells排成行}
        \begin{center}
            \includegraphics[width=0.5\textwidth]{../../pics/integrated-placement-and-routing-for-vlsi-layout-synthesis-and-optimization/stdcell_row.png}
        \end{center}
    \end{exampleblock}
\end{frame}

\begin{frame}
    \frametitle{VLSI Placement Problem}

    \begin{alertblock}{新的问题:macro cells}
        \begin{center}
            \includegraphics[width=0.5\textwidth]{../../pics/integrated-placement-and-routing-for-vlsi-layout-synthesis-and-optimization/macro_cells_placement.png}
            \\
            排成行的macro cells太废面积
        \end{center}
    \end{alertblock}
    \begin{block}{~}
        需要真正的2D placement algorithm
    \end{block}
\end{frame}

\begin{frame}
    \frametitle{VLSI Placement Problem}
    \begin{block}{2D Placement}
        \[
            \min\sum_{i\in\nets} \hpwl(i)
        \]
        s.t., $\forall i,j\in\cells$,
        \begin{align*}
            \left|x_i - x_j\right| &\geqslant \frac{w_i + w_j}{2} \qquad\text{or}\\
            \left|y_i - y_j\right| &\geqslant \frac{h_i + h_j}{2}
        \end{align*}
    \end{block}
    \begin{alertblock}{~}
        然而解得太慢\dots
    \end{alertblock}
\end{frame}

\note{
    \begin{itemize}
        \item 对于cell i,
            \begin{itemize}
                \item $(x_i, y_i)$是cell中心的坐标
                \item $w_i$/$h_i$是其宽度/高度
            \end{itemize}
        \item HPWL(half-perimeter wire length)本身已经是对问题的简化了。
            HPWL是真实线长的下限。
    \end{itemize}
}

\section{The Algorithm}

\begin{frame}
    \frametitle{品山的算法}
    
    \begin{block}{整体框架}
        \begin{enumerate}
            \item 粗放
                \begin{enumerate}
                    \item Gauss-Seidel method on simple quadratic cost
                    \item 2D compaction based on RULD graph
                \end{enumerate}
            \item 细调
                \begin{enumerate}
                    \item 考虑cell的几何形状
                        \begin{enumerate}
                            \item cell shifting
                            \item cell orientation
                        \end{enumerate}
                    \item 2D compaction based on RULD graph
                \end{enumerate}
            \item 精修
                \begin{enumerate}
                    \item 精修HPWL
                    \item 2D compaction based on RULD graph
                \end{enumerate}
        \end{enumerate}
    \end{block}
\end{frame}

\note{
    \begin{itemize}
        \item 每一个步骤都可能导致overlapping。
            所以最后都用2D compaction修一修。
        \item 这里要讲四个算法,依次是2D compaction、initial placement、1st refinement、2nd refinement。
    \end{itemize}
}

\subsection{Compaction}

\begin{frame}
    \frametitle{Compaction}

    \begin{block}{The target and the Name}
        \begin{itemize}
            \item 给定任意一个placement,拍拍紧;如果有重叠,撑撑开。
                \begin{center}
                    \begin{tabu}{lcr}
                        \includegraphics[align=c,width=.3\textwidth]{../../pics/integrated-placement-and-routing-for-vlsi-layout-synthesis-and-optimization/compaction-1.png}
                        &$\Longrightarrow$
                        &\includegraphics[align=c,width=.15\textwidth]{../../pics/integrated-placement-and-routing-for-vlsi-layout-synthesis-and-optimization/compaction-ruld.png}
                    \end{tabu}
                \end{center}
            \item 直觉:横里俄罗斯方块,竖里俄罗斯方块。
                \pause
                \begin{center}
                    \begin{tabu}{lr}
                        \includegraphics[align=c,width=.2\textwidth]{../../pics/integrated-placement-and-routing-for-vlsi-layout-synthesis-and-optimization/compaction-1d.png}
                        &\Large 不行
                    \end{tabu}
                \end{center}
        \end{itemize}
    \end{block}
\end{frame}

\begin{frame}
    \frametitle{Compaction}
    
    \begin{block}{RULD Graph}
        拍紧的同时保持cells之间的\textbf{相对位置}。
        \begin{itemize}
            \item 从任何一个cell(的中心)看出去,周围的cell的角顺序不变。
            \item 任何两个cell直观上的上下、左右关系不变。
        \end{itemize}
    \end{block}
\end{frame}

\begin{frame}
    \frametitle{Compaction}
    
    \begin{exampleblock}{RULD Graph}
        \begin{center}
            \includegraphics[width=.5\textwidth]{../../pics/integrated-placement-and-routing-for-vlsi-layout-synthesis-and-optimization/ruld-graph.png}
        \end{center}
        \begin{itemize}
            \item 每个cell变成一个node。后者的坐标是前者的中心。
            \item 四方边界点
            \item 平面被切割成许多个小三角形 (triangulation)
            \item 除了四方边界点之间,其他边都被赋予\textbf{垂直}或\textbf{水平}的属性。
            \item 每个node都有四向边(legal constraint)。
        \end{itemize}
    \end{exampleblock}
\end{frame}

\begin{frame}
    \frametitle{Compaction}

    \begin{theorem}
        满足以下条件的(合法)RULD图不会有重叠的cells。
        \begin{itemize}
            \item explicit constraints 
                \begin{align*}
                    \left|x_i-x_j\right|\geqslant\frac{w_i+w_j}{2} &&\forall i,j\in\cells\text{~被一条水平线连接}
                \end{align*}
            \item implicit constraints (猜的):
                \\
                对于任意一条垂直边的两端$p,q$,$i$是$p$的右节点,$j$是$q$的左节点,则
                \begin{align*}
                    x_i - x_j \geqslant \frac{w_i + w_j}{2}
                \end{align*}
        \end{itemize}
    \end{theorem}
\end{frame}

\note{
    \begin{itemize}
        \item 还有垂直方向,类似。不赘述。
        \item implicit constraints看了三遍,没get到。
            需要查看更多文献。
        \item 证明的大致思路:以水平方向为例,任意两个cells,以下二者必居其一:
            \begin{enumerate}
                \item 要么他俩同处一条从左边界到右边界由水平边连成的路径上;
                \item 要么他俩分处一条从上边界到下边界由垂直边连成的路径的两边。
            \end{enumerate}
        \item 在该定理的基础上就可以构建使任何placement合法化的算法:
            在构建完RULD graph之后,抛弃所有cells的原有坐标,在explicit/implicit constraints下重建各个cells的坐标。
    \end{itemize}
}

\begin{frame}
    \frametitle{RULD Graph的构建}
    
    \begin{block}{Triangulation}
        \begin{enumerate}
            \item 随便建一个triangulation graph
            \item 每一条边换一换,如果以下目标能改善,则保留。
        \end{enumerate}
        \begin{columns}
            \begin{column}{0.2\textwidth}
                \includegraphics[height=0.5\textheight,width=\textwidth]{../../pics/integrated-placement-and-routing-for-vlsi-layout-synthesis-and-optimization/edge-swapping.png}
            \end{column}
            \begin{column}{0.74\textwidth}
                \small
                \begin{align*}
                    &\min\sum_{(i,j)\in\edges} X_\text{adj}^2(i,j) + Y_\text{adj}^2(i,j)
                    \\
                    X_\text{adj}&=\begin{cases}
                        \frac{2\left|x_i-x_j\right|}{w_i+w_j}    &\text{if }\left|x_i-x_j\right|\leqslant\frac{w_i+w_j}{2}\\
                        \left|x_i-x_j\right|-\frac{w_i+w_j}{2}+1    &\text{otherwise}
                    \end{cases}
                \end{align*}
            \end{column}
        \end{columns}
    \end{block}
\end{frame}

\note{
    \begin{itemize}
        \item 初始建triangulation graph的过程:
            \begin{enumerate}
                \item 四方点先放好。左右连起来。
                \item 每次丢进去一个节点:找到该点的包围三角形,分别连三条边。
            \end{enumerate}
        \item 不理解otherwise部分为什么不直接除。
    \end{itemize}
}

\begin{frame}
    \frametitle{RULD Graph的构建}
    
    \begin{block}{Edge-direction assignment}
        \begin{enumerate}
            \item 对于每一条边,$\pm 45^\circ$标成水平边,其他垂直边。
                \begin{itemize}
                    \item 同一个点的边按四方位(右、上、左、下)排好
                    \item 同一个点不可能有连续两个方位里没有边
                \end{itemize}
            \item 对于每一条边,在保持上述性质的条件下,考虑两段cell的大小。
                \begin{center}
                    \includegraphics[width=0.25\textwidth]{../../pics/integrated-placement-and-routing-for-vlsi-layout-synthesis-and-optimization/critical_slope.png}
                \end{center}
            \item legalization: 保证每个点的四方位都有边
                \quad\includegraphics[align=t,height=0.22\textheight]{../../pics/integrated-placement-and-routing-for-vlsi-layout-synthesis-and-optimization/ruld-legalization.png}
        \end{enumerate}
    \end{block}
\end{frame}

\note{
    \begin{itemize}
        \item legalization:
            有几个patterns,按情况套用。
            \begin{itemize}
                \item 示例图的解释:
                    D上面有个B有个C,A的左面是B右面是C,那么A也肯定在D上面。
            \end{itemize}
    \end{itemize}
}

\begin{frame}
    \frametitle{2D Compaction Based on RULD Graphs}
    
    \begin{exampleblock}{the algorithm}
        \footnotesize
        \begin{algorithmic}[1]
            \Procedure{2dCompact}{$r,t$}
                \Comment $r$: 目标长宽比, $t$: 长宽比的容忍系数
                \State orient $\gets$ HORIZONTAL
                \Loop
                    \Loop
                        \State ratio $\gets$ \Call{AspectRatio}{}
                        \If{ratio $\not\in [\frac{r}{1+t}, (1+t)r]$}
                            \State break
                        \EndIf
                        \State \Call{SelectAndAdjust}{orient}
                    \EndLoop
                    \State \Call{EvalPlacement}{}
                    \If{如果这次的结果比上次差}
                        \State return 上次的结果
                    \EndIf
                    \State \Call{Flip}{orient}
                \EndLoop
            \EndProcedure
        \end{algorithmic}
    \end{exampleblock}
\end{frame}

\note{
    \begin{itemize}
        \item 每次优化一个方向
        \item AspectRatio: 用水平和垂直方向上的critical path的长度除出来。
        \item SelectAndAdjust从一个方向上的critical path里找一条边来调整,
            使得另一个方向上的critical path增加最少。
            \begin{itemize}
                \item 调整的方法只有少数几种local patterns。
                    不需要每次调整都重新算一遍全局RULD。
                    局部改改就行。
                    因此该算法很快。
            \end{itemize}
        \item EvalPlacement是对placement的度量,不同的阶段不一样。
    \end{itemize}
}

\subsection{Initial placement}

\begin{frame}
    \frametitle{Initial placement}
    
    \begin{block}{整体框架}
        \begin{enumerate}
            \item 粗放
                \begin{enumerate}
                    \item Gauss-Seidel method on simple quadratic cost
                    \item \color{gray}2D compaction based on RULD graph
                \end{enumerate}
            \color{gray}
            \item 细调
                \begin{enumerate}
                    \color{gray}
                    \item 考虑cell的几何形状
                        \begin{enumerate}
                            \color{gray}
                            \item cell shifting
                            \item cell orientation
                        \end{enumerate}
                    \item 2D compaction based on RULD graph
                \end{enumerate}
            \item 精修
                \begin{enumerate}
                    \color{gray}
                    \item 精修HPWL
                    \item 2D compaction based on RULD graph
                \end{enumerate}
        \end{enumerate}
    \end{block}
\end{frame}

\begin{frame}
    \frametitle{Initial Placement}
    
    \begin{block}{Simple quadratic formulation}
        \begin{align*}
            F&=\frac{1}{2}\sum_{i,j\in\cells\cup\pads}c_{ij}((x_i-x_j)^2+(y_i-y_j)^2)
            \\\hline
            x_i^{(t+1)}-x_i^{(t)}&=\frac{1}{b_i}\sum_j c_{ij}(x_j^{(t)}-x_i^{(t)})
            \\
            b_i&=\sum_j c_{ij}
        \end{align*}
        \begin{itemize}
            \item 整个算法的一次迭代的时间复杂度$O(cN)$。基本线性。
        \end{itemize}
    \end{block}
\end{frame}

\note{
    \begin{itemize}
        \item $(x_i,y_i)$是cell i的中心点坐标。
        \item 如果没有IO pads,这个公式会把所有cells堆到一起去。
            所以,事先要把IO pads固定下来。
            公式里最后几个$(x_i,y_i)$是IO pads的坐标,不能挪。
        \item $c_{ij}$度量cell i和j之间的连接数。
            但为了避免multiple-pin nets的影响过大,这个系数需要调整。
        \item 一次迭代的时间复杂度里,$c$是平均连接数,$N$是节点数。通常$c$不会很大。
    \end{itemize}
}

\subsection{1st refinement}

\begin{frame}
    \frametitle{1st refinement}
    
    \begin{block}{整体框架}
        \begin{enumerate}
            \item \color{gray}粗放
                \begin{enumerate}
                    \color{gray}
                    \item Gauss-Seidel method on simple quadratic cost
                    \item 2D compaction based on RULD graph
                \end{enumerate}
            \item 细调
                \begin{enumerate}
                    \item 考虑cell的几何形状
                        \begin{enumerate}
                            \item cell shifting
                            \item cell orientation
                        \end{enumerate}
                    \color{gray}
                    \item 2D compaction based on RULD graph
                \end{enumerate}
            \color{gray}
            \item 精修
                \begin{enumerate}
                    \color{gray}
                    \item 精修HPWL
                    \item 2D compaction based on RULD graph
                \end{enumerate}
        \end{enumerate}
    \end{block}
\end{frame}

\begin{frame}
    \frametitle{1st Refinement}

    \begin{block}{考虑macro cell的几何形状}
        \begin{itemize}
            \item macro cells的pins往往不在其中心。
                \begin{itemize}
                    \item pins可以建模成距离其中心的固定偏移。
                \end{itemize}
            \item macro cells的旋转与翻转
                \begin{itemize}
                    \item 可以利用quadratic formulation表现出的引力。
                \end{itemize}
        \end{itemize}
    \end{block}
\end{frame}

\begin{frame}
    \frametitle{1st Refinement}

    \begin{columns}
        \begin{column}{0.47\textwidth}
            \begin{exampleblock}{The algorithm}
                \footnotesize
                \begin{algorithmic}[1]
                    \Loop
                        \State \Call{CellOrient}{}
                        \State \Call{CellShift}{}
                        \State \Call{2dCompact}{}
                        \If{\Call{Terminate?}{}}
                            \State \Call{RestoreBest}{}
                            \State break
                        \Else
                            \State \Call{SaveBest}{}
                        \EndIf
                    \EndLoop
                \end{algorithmic}
            \end{exampleblock}
        \end{column}
        \begin{column}{0.47\textwidth}
            \begin{block}{控制的逻辑}
                \begin{center}
                    \includegraphics[width=\textwidth]{../../pics/integrated-placement-and-routing-for-vlsi-layout-synthesis-and-optimization/quad_cost.png}
                \end{center}
            \end{block}
        \end{column}
    \end{columns}
\end{frame}

\note{
    \begin{itemize}
        \item CellOrient和CellShift总能改善cost,但2dCompact往往反之。
            实验表明,当CellOrient和CellShift的结果和2dCompact的结果大幅收紧,再多跑的意义就不大了。
    \end{itemize}
}

\begin{frame}
    \frametitle{1st Refinement}

    \begin{exampleblock}{The algorithm: cont.}
        \footnotesize
        \begin{algorithmic}[1]
            \Procedure{CellShift}{}
                \State init-hp-cost $\gets$ \Call{EvalHpwl}{}
                \State first-try $\gets$ true
                \Loop
                    \State moves $\gets$ \Call{GaussSeidelMoves}{}
                    \If{first-try}
                        \State moves $\gets$ \Call{ShockMovesOnly}{moves}
                        \Comment 改进2
                        \State first-try $\gets$ false
                    \EndIf
                    \State \Call{MoveCells}{moves}
                    \State new-hp-cost $\gets$ \Call{EvalHpwl}{}
                    \If{new-hp-cost $<$ init-hp-cost $* 0.75$}
                        \Comment 改进1
                        \State break
                    \EndIf
                \EndLoop
            \EndProcedure
        \end{algorithmic}        
    \end{exampleblock}
\end{frame}

\note{
    \begin{itemize}
        \item 改进1:防止过度优化浪费算力。
            根据前面展示的实验的结果,compaction的结果可能大大高于cell shifting。
            所以没必要在一轮迭代里面在illegal placement上花费太多算力。
        \item 改进2:第一次只移动冲击巨大的moves。
            这里“冲击巨大”的定义是:
            对于cell i和j,
            \begin{itemize}
                \item 如果i和j在RULD graph中以一条水平边相连且$x_i^{(t)}<x_j^{(t)}$,则$x_i^{(t+1)}>x_j^{(t+1)}$;
                \item 垂直边类似。
            \end{itemize}
            这里的考虑是
            \begin{itemize}
                \item 如果不破坏RULD graph,那么后面的compaction通常会把cell推回原来的位置(附近)。
                \item 只有第一次的原因是:shock moves破坏了RULD graph,重建一遍太费事。
            \end{itemize}
    \end{itemize}
}

\subsection{2nd refinement}

\begin{frame}
    \frametitle{2nd refinement}
    
    \begin{block}{整体框架}
        \begin{enumerate}
            \color{gray}
            \item 粗放
                \begin{enumerate}
                    \color{gray}
                    \item Gauss-Seidel method on simple quadratic cost
                    \item 2D compaction based on RULD graph
                \end{enumerate}
            \item 细调
                \begin{enumerate}
                    \color{gray}
                    \item 考虑cell的几何形状
                        \begin{enumerate}
                            \color{gray}
                            \item cell shifting
                            \item cell orientation
                        \end{enumerate}
                    \item 2D compaction based on RULD graph
                \end{enumerate}
            \item 精修
                \begin{enumerate}
                    \item 精修HPWL
                    \color{gray}
                    \item 2D compaction based on RULD graph
                \end{enumerate}
        \end{enumerate}
    \end{block}
\end{frame}

\begin{frame}
    \frametitle{2nd Refinement}

    \begin{exampleblock}{从quadratic cost换成HPWL}
        \footnotesize
        \begin{algorithmic}[1]
            \For{循环三遍}
                \State \Call{SwapNeighbours}{}
                \State \Call{Orientate}{}
                \State \Call{2dCompact}{}
            \EndFor
            \State 返回最后的结果
        \end{algorithmic}
    \end{exampleblock}
\end{frame}

\note{
    \begin{itemize}
        \item SwapNeighbours
            \begin{itemize}
                \item 对于每个cell,在其RULD graph四跳或六跳之内再找一个cell,交换。
                    如果能改善HPWL,则保留。
                \item 这个操作不会改变RULD graph。
            \end{itemize}
        \item Orientate
            \begin{itemize}
                \item 对于每个cell,原地尝试其8个方向,如果能改善HPWL则保留。
                \item 这个操作也不会改变RULD graph。
            \end{itemize}
        \item HPWL是否比quadratic cost更优,学术界有争议。
            HPWL实质是manhattan distance,而quadratic cost在优化Euclidean distance。
            Harry认为后者优化出来的结果更接近正方形,这对routing有好处。
    \end{itemize}
}

\end{document}
