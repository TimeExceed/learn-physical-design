\documentclass[UTF8,a4paper]{ctexart}
\usepackage[lmargin=2cm,rmargin=2cm,tmargin=2cm,bmargin=2cm]{geometry}
\usepackage{graphicx}
\usepackage{minted}
\usepackage{amsmath}
\usepackage{amssymb}
\usepackage{amsthm}
\usepackage{awesomebox}
\usepackage{tabu}
\usepackage{subfig}
\usepackage{hyperref}

\title{Fence-Region-Aware Mixed-Height Standard Cell Legalization\\
    读书笔记}
\author{陶大}
\date{Dec. 2021}

\renewcommand{\emph}[1]{\textbf{#1}}
\renewcommand{\leq}{\leqslant}
\renewcommand{\atop}[2]{\genfrac{}{}{0pt}{}{#1}{#2}}
\newtheorem{theorem}{Theorem}
\newtheorem{definition}{Definition}

\begin{document}
\maketitle
\tableofcontents

\section{解决的问题}

多行std cell的detail placement。

\section{步骤}

见图\ref{steps}。

\begin{figure}
    \begin{center}
        \includegraphics[width=\textwidth]{./whole-picture.png}
    \end{center}
    \caption{步骤}        
    \label{steps}
\end{figure}

\section{Pre-legalization}

Global placement 有可能将cells放在不能放的区域。
本步骤将错放的cells推到最近的可以放的区域,不考虑overlapping。

\section{Cell arrangement}

\begin{itemize}
\item 对于每个cell,以global placement的结果为起点,用FRA-BFS算法找到最近合适的位置,放下。
\item 实现中,先把多行cell放掉,再放单行cell。
\item FRA-BFS算法只处理有限大小的领域,所以还是有可能有些cell找不到地方放。
\end{itemize}

\section{Local cell shifting}

\begin{figure}
    \begin{center}
        \subfloat{%
            \includegraphics[width=.5\textwidth]{local-cell-shifting-ab.png}%
            \label{local-cell-shifting:a}}
        \qquad
        \subfloat{%
            \includegraphics[width=.25\textwidth]{local-cell-shifting-d.png}%
            \label{local-cell-shifting:b}}
    \end{center}
    \caption{Local cell shifting}
    \label{local-cell-shifting}
\end{figure}

对于上一步骤没有放下的cell P,
\begin{enumerate}
\item 看P的一个领域(图\ref{local-cell-shifting:a} 左图)。
    把跨过边界的cells认为不可移动(图\ref{local-cell-shifting:a}右图深色块),
    其他cells认为可以移动(图\ref{local-cell-shifting:a}右图浅色块)。
\item 把可移动块(a-f)先拿走。
\item 把P放下。
    再依次用FRA-BFS算法把a-f放下(论文中说按大小降序排列,OpenROAD实现成随意顺序)。
    得到图\ref{local-cell-shifting:b}。
\end{enumerate}
但是可能global placement在某些区域的密度太高,以至于无法在有限领域内解决。

\section{Corner weight move}

\begin{cautionblock}
    没看懂。
    代码里没实现。
\end{cautionblock}

\section{Postoptimization}

以cell-move和cell-swap为手段,以总错位值最小为优化目标,进行模拟退火。

\begin{noteblock}
    错位是指global placement放的位置是 $P$,legalization之后放在 $P'$。
    那么错位值是 $\left|P'_x-P_x\right| + \left|P'_y-P_y\right|$。
\end{noteblock}

\begin{noteblock}
    代码里没有实现。
    仅实现了随机交换。

    实现的代码也没有用起来。
\end{noteblock}

\subsection{Cell move}

在前面legalization的过程中,一个cell被放置在位置 $P'$。
在固定其他cells的位置的条件下,在 $P'$ 的领域里重新找个位置 $P''$。
如果 $P''$ 改进了 $P'$ 的错位,则替换。

\begin{noteblock}
    论文里没有讲清楚进行此项优化的条件。
    查代码,是指 $P''$ 距离 $P$ 足够近。
\end{noteblock}

\subsection{Cell swap}

交换两个形状一样的cells。
因为形状一样,所以不会改变layout。
如果能减少整体的错位,就保留。

\section{FRA-BFS算法}

\begin{figure}
    \begin{center}
        \begin{minted}[autogobble]{python}
            def fra_bfs(g, c, p):
                '''
                g: 已经放了一些cells的布局图
                c: 准备放下的cell
                p: 开始搜索的位置
                '''
                a = [] # 可以放下c的位置
                s = INIT_WINDOW_SIZE
                while True:
                    if s超过阈值:
                        return None
                    h = 从g构造以p为中心s为范围的窗口
                    for t in h内可以放置c的位置:
                        a.append(t)
                    if len(a) > 0:
                        return a中距离p最近的位置t
                    增大s
        \end{minted}
    \end{center}
    \caption{FRA-BFS算法}
    \label{fra-bfs}
\end{figure}

算法见\ref{fra-bfs}。
\begin{itemize}
\item 从g构造h的方法如图 \ref{local-cell-shifting:a}。
\item 由于s一点一点增大,实际的效果就是由近及远一圈一圈找合适的位置。
\item 把s的阈值去掉,可以处理全图。
\end{itemize}

\begin{noteblock}
    代码里s是hard code的3。
\end{noteblock}

\section{评述}

密度对算法的影响体现在两个方面。
\begin{description}
\item[速度的方面] 
    如果密度不太高,那么在小领域就能把目标cell放下。
    算法很快就能结束。
    并且可以并行处理多个待放的cell,只要它们的领域不重叠。
\item[效果的方面]
    如果密度高,那么随机涨落之下,有些区域就是不足以放下所有global placement丢进去的cells。
    \begin{itemize}
    \item 论文说用 corner weight move 来解决。
        一来没看懂,二来没代码。
        不知道怎么搞。
    \item 把FRA-BFS中s的阈值去掉,一个迭代内轮流使用cell arrangement和local cell shifting可以处理全图。
        但这样有可能把一个cell推到非常远的位置。
        想象一下两种做法:
        \begin{enumerate}
        \item 把一个cell丢到很远,其他cell待在原地。
        \item 把cells一圈一圈往外推。
        \end{enumerate}
        很难绝对地说何者更优。
        直观上似乎后者适应更多的designs。
    \end{itemize}
\item 直觉上,postoptimization并不能把前者修成后者。
\end{description}

\end{document}
