\documentclass[UTF8,a4paper]{ctexart}
\usepackage[lmargin=2cm,rmargin=2cm,tmargin=2cm,bmargin=2cm]{geometry}
\usepackage{graphicx}
\usepackage{minted}
\usepackage{amsmath}
\usepackage{amssymb}
\usepackage{amsthm}
\usepackage{awesomebox}
\usepackage{hyperref}

\title{Almost Optimum Placement Legalization by Minimum Cost Flow and Dynamic Programming\\
    读书笔记}
\author{陶大}
\date{Sep. 2021}

\renewcommand{\emph}[1]{\textbf{#1}}
\renewcommand{\leq}{\leqslant}
\newcommand{\supp}[1]{\text{supp} ({#1})}
\newcommand{\dem}[1]{\text{dem} ({#1})}
\renewcommand{\atop}[2]{\genfrac{}{}{0pt}{}{#1}{#2}}
\newcommand{\bal}{\text{bal}}
\newcommand{\lack}{\text{lack}}
\newtheorem{theorem}{Theorem}
\newtheorem{definition}{Definition}

\begin{document}
\maketitle
\tableofcontents

\section{解决的问题}

在单行std cell的条件下,高速地做到接近最优的detail placement。

\section{步骤}

\begin{minted}[autogobble,frame=single]{python}
    assignment = assign每个cell到其中心所在的region
    while assignment is illegal:
        f = 构造图并计算最小费用最大流(assignment)
        assignment = 将f实现成cell-region assignment
    placement = 诱导(assignment)
    postoptimize(placement)
\end{minted}

\section{最大流图}

\begin{figure}
    \begin{center}
        \includegraphics[width=0.5\textwidth]{./concepts.png}    
    \end{center}
    \caption{Zone \& Region}
\end{figure}

\begin{definition}
    \;
    \begin{description}
    \item[zone] 
        一个zone是从blockade或边界到blockade或边界连续的一行。
    \item[region]
        一个zone可以继续被切分成多个regions。
        论文里说 region 都是等宽的,是zone和chip-wide columns相交的方块。
        但算法里似乎并不需要这一假设。
    
        论文里以 $A_i$ 代表region $i$,其宽度为 $w(A_i)$。
        $C_i=\{c^i_1,\dots,c^i_k\}$ 为assign给该region的所有cells,从左往右一字排开。
        其中 $c^i_j$ 的宽度是 $w(c^i_j)$。
        并且简记 $w(C_i)=\sum_j w(c^i_j)$。
    
        \begin{noteblock}
            region可以包含多个工艺要求的grids。
            这样一方面可以减少需要处理的节点数量。
            另外论文里假设任何cell最多只会和两个region有重叠(即跨过一次region-region边界)。
            
            所以,实现中需要考虑怎样确定region的宽度。
            论文里没讲怎么做。
        \end{noteblock}
    \item[Soft boundary feasible, SB-feasible]
        在一个assignment诱导的placement里,如果所有cells的中心都在其assign的region内,
        则称此assignment为SB-feasible。
    \item[Hard boundary feasible, HB-feasible]
        在SB-feasible的基础上额外约束所有cells不跨region的边界。
    
        \begin{noteblock}
            显然hard boundary的约束更强,会导致更多的cell移动。
            这里不再展开如何在HB约束下如何构造最大流图。
            关心其中细节的请去看论文。
        \end{noteblock}
    \item[Interval]
        假设一个zone内从左往右有 $\{A_1,\dots,A_k\}$。
        那么对于任意$1\leq \mu \leq \nu \leq k$,
        定义Interval及其宽度为
        \begin{align*}
            A_{\mu,\nu} &=\{A_\mu,\dots,A_\nu\}\\
            w(A_{\mu,\nu}) &=\sum_{j=\mu}^{\nu} w(A_j)
        \end{align*}
        而一个region可以视作退化的Interval。
    \item[SB约束下的供应和需求]
        对于任意Interval $A_{\mu,\nu}$,定义其供应为
        \begin{align*}
            s_{\mu,\nu} &:=\max\biggl\{%
                0,\;%
                \sum_{i=\mu}^{\nu}\bigl(w(C_i)-w(A_i)\bigr)%
                -\frac{1}{2}\bigl(w(c^{\mu}_1)+w(c^{\nu}_{k_\nu})\bigr)%
            \biggl\}
            \\
            \supp{A_{\mu,\nu}} &:=\max\biggl\{%
                0,\;%
                s_{\mu,\nu}%
                -\sum_{\atop{\mu\leq\mu'\leq\nu'\leq\nu}{(\mu,\nu)\neq(\mu',\nu')}}\supp{A_{\mu',\nu'}}%
            \biggr\}
        \end{align*}
        其中 $c^\mu_1$ 是指 region $A_\mu$ 里最左边的cell。
        类似地, $c^\nu_{k_\nu}$ 是指 region $A_{\nu}$ 里最右边的一个cell。
    
        我们称 $A_{\mu,\nu}$ 是一个 \emph{供应interval},如果 $\supp{A_{\mu,\nu}}>0$ 。
        所谓供应,供应的是放不下的cells。
        $s_{\mu,\nu}$定义了 \emph{interval的过载},
        意思是 $A_{\mu,\nu}$ 只能容纳这么多cells(以宽度计),
        现在assign了这么多的cells
        (因为SB constraints,左右两端的cells都只能算半个)。
        \begin{itemize}
        \item 一方面,可能一个interval过载而其中每一个subinterval都不过载。
            比如,一个由两个regions组成的interval。
            其中左region伸了半个cell到右region,而右region又没这么多的余地给它。
            这种情况,由于SB constraints,两个region自身都没有过载,而整体的interval过载。
        \item 另一方面,一个过载的interval可能是由于其中一个局部过载,
            即,从这个局部挪走一个cell,可能一把interval都不过载了。
            这个局面也不是我们想要的。
            因此 $\supp{\cdot}$ 把子段的供应都扣除。
        \end{itemize}
        
        类似地,我们定义 \emph{空载 $t(\cdot)$ 和需求 $\dem{\cdot}$}。
        \begin{align*}
            t_{\mu,\nu} &=\min\biggl\{%
                0,\:%
                \sum_{i=\mu}^{\nu}\bigl(w(C_i)-w(A_i)\bigr)%
                +\frac{1}{2}\bigl(w(c^{\mu-1}_{k_{\mu-1}})+w(c^{\nu+1}_1)\bigr)
            \biggr\}
            \\
            \dem{A_{\mu,\nu}} &=\min\biggl\{%
                0,\;%
                t_{\mu,\nu}%
                -\sum_{\atop{\mu\leq\mu'\leq\nu'\leq\nu}{(\mu,\nu)\neq(\mu',\nu')}}\dem{A_{\mu',\nu'}}%
            \biggr\}
        \end{align*}
    \item[Soft boundary minimum cost flow, SB-MCF]
        构造一个图,其中节点
        \begin{itemize}
        \item 所有的interval都是节点(因此所有的region都是节点)。其中
            \begin{itemize}
            \item $\supp{\cdot}>0$ 称为供应节点
            \item $\dem{\cdot}<0$ 称为需求节点
            \end{itemize}
        \item 增加 $s,t$ 两个虚节点。
        \end{itemize}
        以及边
        \begin{itemize}
        \item 对于相邻的两个regions $A_i,A_j$,任意 $c^i_\alpha\in A_i$,
            从 $A_i$ 到 $A_j$ 拉一条有向边,
            \begin{itemize}
            \item 其capacity为 $w(c^i_\alpha)$
            \item 其cost为 $\frac{\cdot}{w(c^i_\alpha)}$。
                空穴的部分是将$c^i_\alpha$ 从 $A_i$ 的中心挪到 $A_j$ 的中心的线长差(后者减前者)。
            \end{itemize}
            \begin{noteblock}
                论文里没讲什么是相邻。
            \end{noteblock}
        \item 从 $s$ 向所有供应节点 $A_{\mu,\nu}$ 拉一条有向边,
            \begin{itemize}
            \item 其capacity为 $\supp{A_{\mu,\nu}}$
            \item 其cost为0
            \end{itemize}
            类似地,从所有需求节点向 $t$ 拉一条有向边。
        \item 对于任意供应interval $A$ 和任意一个节点 $A'\subset A$,
            拉一条从 $A$ 到 $A'$ 的有向边,其cost为0而capacity无限。
    
            类似地处理需求intervals。
            \begin{noteblock}
                从论文的描述来看,可以从非平凡interval到非平凡interval拉边。
                但我没想到这样做的好处。
            \end{noteblock}
        \item 从 $t$ 拉有向边到所有的intervals,其cost为0而capacity无限。
            \begin{noteblock}
                我的理解,只有在边cost可能为负的情况下此设置才有意义。
                如果不对初始assignment做任何假设,这是有可能的。
                如果优化目标不是线长,这也是有可能的。
            \end{noteblock}
        \end{itemize}
    \item[interior movement]
        在SB约束下,移动一个interval两边的cell相比移动其他的cell情况更复杂。
        我们将移动非两边的cell称为 interior movement。
    \end{description}
\end{definition}

直观来讲,沿着最大流图的一条从$s$走向$t$的流,
\begin{itemize}
\item 如果其中一步从region $A_i$ 到相邻region $A_j$,意味着将一个cell从 $A_i$ 推到 $A_j$。
\item 从$s$到供应节点 $A_i$意味着,$A_i$中可以有cell(s)挪走。
\item 从非平凡interval到region走一步意味着,从该region挪走一个cell可以改善该interval的过载。
\end{itemize}
论文中讲以上直觉形式化成如下定理。

\begin{theorem}
    非平凡供应intervals的数量不会超过regions数量的一倍。
    非平凡需求intervals的情况同样。
\end{theorem}

\begin{theorem}
    令 $f$ 是一个SB-MCF的解。
    如果通过interior movement可以恰好实现 $f$,那么该实现是一个 SB-feasible assignment。
\end{theorem}

\begin{theorem}
    假设一个SB-MCF实例减少了一个供应量,且令 $f$ 是修改后的实例的一个解。
    那么 $f$ 的任何通过interior movement的恰好实现都不是SB-feasible assignment。
\end{theorem}

\begin{theorem}
    假设供应之和大于任何单个zone($\max_{z\in\text{zones}}\{w(A_z)-w(C_z)\}$)。
    减少SB-MCF实例的任意数量的需求,且$f$是修改后实例的任意最大流。
    那么 $f$ 的任何通过interior movement的恰好实现都不是SB-feasible assignment。
\end{theorem}

\begin{theorem}
    令$f$是一个SB-MCF的最优解。
    如果$f$任意近似实现对于任意interval $A_{\mu,\nu}$,如果有
    \begin{align*}
        \bal(A_{\mu,\nu}) &:=\text{total width of leaving cells} - \text{total width of entering cells}\\
        \bal(A_{\mu,\nu}) &\geq \sum_{i=\mu}^{\nu}\bigl(%
            w(C_i)-w(A_i)%
        \bigr)%
        -\frac{1}{2}\bigl(%
            w(c^{\mu}_1)+w(c^{\nu}_{k_\nu})
        \bigr)%
    \end{align*}
    那么该近似实现最多将cells水平移动一个region。
\end{theorem}

\section{最大流图的实现}

\begin{definition}
    给定一张图$G$与其上一个流 $f$。
    对于任意$v\in G$,其入边记为 $\delta^-(v)$,其出边记为 $\delta^+(v)$。
    相应的流量 
    \begin{align*}
        f(\delta^-(v)) &:=\sum_{e\in\delta^-(v)}f(e)\\
        f(\delta^+(v)) &:=\sum_{e\in\delta^+(v)}f(e)
    \end{align*}
\end{definition}

整个算法从去除了intervals, $s$ and $t$的子图开始。
\begin{enumerate}
\item 从子图里挑选一个节点$v$,满足 $f(\delta^-(v))=0$ 且 $f(\delta^+(v))>0$。
    设子图里$v$的相邻点为 $X$。
\item 寻找一个partition $p$,
    \begin{align*}
        p_v:\;&C_v\to X\cup\{v\}\\
        p_{x}:\;& C_{x}\to \{x,v\}\qquad\forall x\in X
    \end{align*}
    s.t.
    \begin{align}
        \sum_{y\in X\cup\{v\}}\sum_{\atop{c\in C_y}{p_y(c)=v}}w(c)
        \:-\:
        \sum_{c\in C_v}w(c)
        \:+\:
        f(\delta^+(v))
        &\leq\lack(v)
        \label{part:eq:v}
        \\
        \sum_{\atop{c\in C_v}{p_v(c)=x}}w(c)
        \:-\:
        \sum_{\atop{c\in C_x}{p_x(c)=v}}w(c)
        \:-\:
        f((v,x))
        &\leq\text{lack(x)}\quad\forall x\in X
        \label{part:eq:neighbour}
    \end{align}
    其中 $\lack(v)$ 是节点$v$可以额外接受的流量,即,
    \[
        \lack(v)=\begin{cases}
            -\dem{v}-f((v,t)),
            &\text{for all demand node }v
            \\
            0,
            &\text{otherwise}
        \end{cases}
    \]
    定义$p$的cost是其中移动的cells的cost之和。
    那么寻找代价最小的 $p$ 类似一个0-1背包问题,可以用一个动态规划搞定。
\item ~
    \begin{itemize}
    \item 如果找到了那么一个partition,则取用。
    \item 否则,我们找到满足\ref{part:eq:v}并且violate \ref{part:eq:neighbour}最少的partition。
        从每个violation节点$x$到$t$在子图里的最短路径一路增加$\lack$,直到满足\ref{part:eq:neighbour}。
    \item 如果还是不行,我们找到violate \ref{part:eq:v}和\ref{part:eq:neighbour}最少的partition,
        并进行上述修复。
    \end{itemize}
    按照该partition移动cells,并更新 $f$,将$v$的出边的流量降为0,并且从子图中移除 $v$。
\end{enumerate}
如果最终的结果并不SB feasible,则增大一倍region的width,再次迭代。

\begin{noteblock}
    有两点不理解
    \begin{itemize}
    \item $f(\delta^-(v))=0$意味着没有cell需要搬入$v$,那为何 $p$ 却允许如此做?
    \item $p$ 的构造只考虑了相邻关系,而没有考虑流量。
        是否意味着,$p$ 会沿着无流量的边移动cell呢?
    \end{itemize}
\end{noteblock}

\section{Single-zone placement}

即使给定一个zone内所有的cells,以最小化cost为目标来摆放,这个问题仍然NP-hard。
因此我们进一步简化:假定其中所有cells的顺序不变。
论文给了该问题的$O(n)$解法的引用。

\section{Postoptimization}

\begin{cautionblock}
    to be filled
\end{cautionblock}

\section{Lower bounds}

\begin{cautionblock}
    to be filled
\end{cautionblock}

\end{document}
